% !TEX TS-program = lualatex

\documentclass[lualatex]{beamer}
\usepackage{hyperref}
\usepackage{fontspec}
\usepackage{fontawesome}
\usepackage{mathtools}
\usepackage{mathabx}
\usepackage{relsize}
\renewcommand{\_}{\textscale{.6}{\textunderscore}}

\usetheme{Boadilla}
\usecolortheme{beaver}
%% the metropolis theme is not compatible with fontawesome package
%% \usetheme[progressbar=frametitle]{metropolis}

\usepackage{appendixnumberbeamer}

\usepackage{booktabs}
\usepackage[scale=2]{ccicons}

\usepackage{pgfplots}
\usepgfplotslibrary{dateplot}

\usepackage{xspace}
\newcommand{\themename}{\textbf{\textsc{metropolis}}\xspace}

% turn off figures and tables numberings
\usepackage{caption}
\captionsetup[figure]{labelformat=empty}
\captionsetup[table]{labelformat=empty}

%source code
\usepackage{listings}
\lstset{
  language=Haskell,
  basicstyle=\fontsize{11}{13}\ttfamily,
  breaklines=true,
  keepspaces=true,
  numbersep=5pt,
  showspaces=false,
  columns=fullflexible,
  mathescape
}

%images
\usepackage{graphicx}

%commutative diagrams
\usepackage{tikz-cd}

\usepackage{color}

\begin{document}

\title[fold tutorial]{A tutorial on the universality and expressiveness of fold}
\subtitle{by Graham Hutton}
\date[May 2018]{Papers We Love. Mad \\ Madrid, Spain}
\author[A. Trunov \& J. Domínguez]{narrated by Anton Trunov \& Jesús Domínguez}
\institute[IMDEA]{IMDEA Software Institute \\ Madrid, Spain}


\maketitle

%%%%%%%%%%%%%%%%%%%%%%%%%%%%%%%%%%%%%%%%%%%%%%%%
\section{Who we are}
%%%%%%%%%%%%%%%%%%%%%%%%%%%%%%%%%%%%%%%%%%%%%%%%

\begin{frame}[fragile]{Anton Trunov}
I'm a research programmer at \href{https://software.imdea.org}{IMDEA Software Institute} \\
\includegraphics[width=107mm]{IMDEA_Software.jpg}
\vspace{5mm}
\renewcommand{\arraystretch}{1.5}
\begin{tabular}{l c r}
  {\Large {\color{orange} \faStackOverflow}} \href{https://www.stackoverflow.com/users/2747511/anton-trunov}{anton-trunov} &
  {\Large {\color{cyan} \faTwitter}} \href{https://twitter.com/Anton_A_Trunov}{@Anton\_A\_Trunov} &
  {\Large {\color{black} \faGithub}} \href{https://www.github.com/anton-trunov}{anton-trunov}
\end{tabular}
\end{frame}

\begin{frame}[fragile]{Jesús Domínguez}
I'm a PhD student at \href{https://software.imdea.org}{IMDEA Software Institute}
% Maybe it is better to remove this picture already?
%\includegraphics[width=107mm]{IMDEA_Software.jpg}

\vspace{5mm}

My interests include:
\begin{itemize}
\item Type Theory
\item Logic
\item Concurrency
\item Semantics of programming languages
\end{itemize}

\vspace{5mm}

\renewcommand{\arraystretch}{1.5}
\begin{tabular}{l c r}
  {\Large {\color{orange} \faStackOverflow}} \href{https://stackoverflow.com/users/9714364/jes%c3%bas-dom%c3%adnguez}{jesus-dominguez} &
  {\Large {\color{cyan} \faTwitter}} \href{https://twitter.com/jeshecdom}{@jeshecdom} &
  {\Large {\color{black} \faGithub}} \href{https://github.com/jeshecdom}{jeshecdom}
\end{tabular}
\end{frame}

\begin{frame}[fragile]{Why we love the paper}
\Large
\begin{itemize}
\item Highly readable
\item Starts with the basics
\item Teaches how to calculate functions' \emph{implementations}
\item Good overview of the area
\end{itemize}
\end{frame}

%%%%%%%%%%%%%%%%%%%%%%%%%%%%%%%%%%%%%%%%%%%%%%%%
\section{Motivational examples}
%%%%%%%%%%%%%%%%%%%%%%%%%%%%%%%%%%%%%%%%%%%%%%%%

\begin{frame}[fragile]{Can you see the pattern here?}
\begin{columns}
\column{0.45\textwidth}
\begin{lstlisting}[language=Haskell]
sum :: [Int] -> Int
sum [] = 0
sum (x:xs) = x + (sum xs)
\end{lstlisting}
\pause
\begin{lstlisting}[language=Haskell]
all :: [Bool] -> Bool
all [] = True
all (x:xs) = x && (all xs)
\end{lstlisting}
\pause
\column{0.55\textwidth}
\begin{lstlisting}[language=Haskell, morekeywords={foldr}]
length :: [a] -> Int
length [] = 0
length (_:xs) = 1 + (length xs)
\end{lstlisting}
\pause
\begin{lstlisting}[language=Haskell, morekeywords={foldr}]
map :: (a -> b) -> [a] -> [b]
map f [] = []
map f (x:xs) = (f x) : (map f xs)
\end{lstlisting}
\end{columns}
\begin{columns}
\column{0.5\textwidth}
\begin{center}
% f and v are free variables here
\pause
\begin{lstlisting}[language=Haskell, morekeywords={rec}]]
rec :: [a] -> b
rec [] = v
rec (x:xs) = x `f` (rec xs)
\end{lstlisting}
\end{center}
\end{columns}
\end{frame}

\begin{frame}[fragile]{More examples of the same pattern}
\begin{lstlisting}[language=Haskell, morekeywords={foldr}]
reverse :: [a] -> [a]
reverse [] = []
reverse (x:xs) = x `snoc` (reverse xs)
  where snoc x xs = xs ++ [x]
\end{lstlisting}
\pause
\begin{lstlisting}[language=Haskell, morekeywords={foldr}]
filter :: (a -> Bool) -> [a] -> [a]
filter p [] = []
filter p (x:xs) = x ?: (filter p xs)
  where (?:) x xs = if p x then x : xs else xs
\end{lstlisting}
\pause
\begin{lstlisting}[language=Haskell, morekeywords={foldr}]
(++) :: [a] -> [a] -> [a]
(++) [] ys = ys
(++) (x:xs) ys = x : ((++) xs ys)
\end{lstlisting}
\end{frame}

%%%%%%%%%%%%%%%%%%%%%%%%%%%%%%%%%%%%%%%%%%%%%%%%
\section{Definition of foldr}
%%%%%%%%%%%%%%%%%%%%%%%%%%%%%%%%%%%%%%%%%%%%%%%%

\begin{frame}[fragile]{Abstracting out the pattern}
% in Haskell, the real type of foldr is
% Foldable t => (a -> b -> b) -> b -> t a -> b
% but we would not mention it during the talk
\begin{lstlisting}[language=Haskell, morekeywords={foldr}]
foldr :: (a -> b -> b) -> b -> [a] -> b
foldr f v [] = v
foldr f v (x:xs) = x `f` (foldr f v xs)
\end{lstlisting}
\end{frame}

%% We are going to concentrate on finite lists only
\begin{frame}[fragile]{What does fold do?}
\begin{lstlisting}[language=Haskell, morekeywords={foldr}]
foldr ($\oslash$) v [x$_1$, x$_2$, x$_3$, ..., x$_{n-1}$ x$_n$]
  =
x$_1$ $\oslash$ (x$_2$ $\oslash$ (x$_3$ $\oslash$ ... (x$_{n-1}$ $\oslash$ (x$_n$ $\oslash$ v)) ... ))
\end{lstlisting}
% Notice that the parentheses associate to the right,
% hence the standard Haskell name for fold -- foldr
\vspace{5mm}
Compare with
\begin{lstlisting}[language=Haskell, morekeywords={foldr}]
x$_1$ : (x$_2$ : (x$_3$ : ... (x$_{n-1}$ : (x$_n$ : [])) ... ))
\end{lstlisting}
\pause
\begin{itemize}
\item{\texttt{foldr} ``replaces'' (\texttt{:}) with $\oslash$}
\item{and \texttt{[]} with \texttt{v}}
\end{itemize}
\end{frame}

\begin{frame}[fragile]{Let's reimplement the functions we already saw}
\begin{lstlisting}[language=Haskell, morekeywords={foldr}]
sum = foldr (+) 0
x$_1$ + (x$_2$ + (x$_3$ + ... (x$_{n-1}$ + (x$_n$ + 0)) ... ))
\end{lstlisting}
\pause
\begin{lstlisting}[language=Haskell, morekeywords={foldr}]
all = foldr (&&) True
x$_1$ && (x$_2$ && (x$_3$ && ... (x$_{n-1}$ && (x$_n$ && True)) ... ))
\end{lstlisting}
\pause
\begin{lstlisting}[language=Haskell, morekeywords={foldr}]
length = foldr (const (1+)) 0
1 + (1 + (1 + ... (1 + (1 + 0)) ... )))
\end{lstlisting}
\pause
\begin{lstlisting}[language=Haskell, morekeywords={foldr}]
map f = foldr (\x r -> f x : r) []
f x$_1$ : (f x$_2$ : (f x$_3$ : ... (f x$_{n-1}$ : (f x$_n$ : [])) ... ))
\end{lstlisting}
\end{frame}

\begin{frame}[fragile]{Let's reimplement the functions we've already seen}
\begin{lstlisting}[language=Haskell, morekeywords={foldr}]
reverse = foldr snoc []
  where snoc x xs = xs ++ [x]
x$_1$ `snoc` (x$_2$ `snoc` ... (x$_{n-1}$ `snoc` (x$_n$ `snoc` [])) ... )
\end{lstlisting}
\pause
\begin{lstlisting}[language=Haskell, morekeywords={foldr}]
filter p (x:xs) = foldr (?:) []
  where (?:) x xs = if p x then x : xs else xs
x$_1$ ?: (x$_2$ ?: (x$_3$ ?: ... (x$_{n-1}$ ?: (x$_n$ ?: [])) ... ))
\end{lstlisting}
\pause
%% (++) has actually different shape
\begin{lstlisting}[language=Haskell, morekeywords={foldr}]
(++) xs ys = foldr (:) ys xs
-- or, in the point-free style: (++) = flip (foldr (:))
x$_1$ : (x$_2$ : (x$_3$ : ... (x$_{n-1}$ : (x$_n$ : ys)) ... ))
\end{lstlisting}
\end{frame}

\begin{frame}[fragile]{Advantages of using fold}
\Large
\begin{itemize}
\item{Less boilerplate -- we don't repeat the recursion scheme}
\item{Can be easier to understand -- the essence of the algorithm can be seen clearer} %% (once you build intuition on how fold works)
\item{The code can be constructed in a systematic manner}
\item{Easier code transformations}
\item{Makes it easier to prove the properties of functions}
\end{itemize}
\end{frame}

%%%%%%%%%%%%%%%%%%%%%%%%%%%%%%%%%%%%%%%%%%%%%%%%
\section{dropWhile example}
%%%%%%%%%%%%%%%%%%%%%%%%%%%%%%%%%%%%%%%%%%%%%%%%

%dropWhile is a primitive recursive function.
%So, is fold weaker???
%fold is powerful enough to encode any primitive recursive function
%(as long as your language has pair types),

\begin{frame}[fragile]{fold is \emph{not} a silver bullet}

\begin{lstlisting}[language=Haskell, morekeywords={foldr}]
dropWhile :: (a -> Bool) -> [a] -> [a]
dropWhile p [] = []
dropWhile p (x:xs) = if p x then dropWhile p xs else x : xs
\end{lstlisting}
\begin{lstlisting}[language=Haskell, morekeywords={foldr}]
-- example:
dropWhile even [4,2,1,2,3,4,5] = [1,2,3,4,5]
\end{lstlisting}
\pause
The folding function \texttt{f} gets
\begin{itemize}
\item{the head of the list \texttt{x}}
\pause
\item{the result of calling \texttt{fold} on tail}
\pause
\[
\underbrace{x_1}_\text{head} \oslash
\underbrace{(x_2 \oslash (x_3 \oslash ... (x_{n-1} \oslash (x_n \oslash v)) ... ))}_\text{foldr f v [$x_2$, ..., $x_n$]}
\]
\pause
\item{but not the tail itself!}
\end{itemize}
\end{frame}
%% So, let's see how we can implement tail using fold
\begin{frame}[fragile]{fold is \emph{not} a silver bullet}
\begin{lstlisting}[language=Haskell, morekeywords={foldr}]
tail :: [a] -> [a]
tail [] = []          -- Exception in Haskell
tail (_:xs) = xs
\end{lstlisting}
\pause
% this is basically Kleene's definition of the predecessor function for the Church numerals
\begin{lstlisting}[language=Haskell, morekeywords={foldr}]
tail :: [a] -> [a]
tail = snd . foldr (\x (xs, acc) -> (x : xs, xs)) ([], [])
\end{lstlisting}
\pause
Example:
\begin{lstlisting}[language=Haskell, morekeywords={foldr}]
foldr (\x (xs, acc) -> (x : xs, xs)) ([], []) [$x_1$, $x_2$, $x_3$]
\end{lstlisting}\pause
\begin{lstlisting}[language=Haskell, morekeywords={foldr}]
[] ==> ([], [])
\end{lstlisting}\pause
\begin{lstlisting}[language=Haskell, morekeywords={foldr}]
x$_3$ ==> ([x$_3$], [])
\end{lstlisting}\pause
\begin{lstlisting}[language=Haskell, morekeywords={foldr}]
x$_2$ ==> ([x$_2$, x$_3$], [x$_3$])
\end{lstlisting}\pause
\begin{lstlisting}[language=Haskell, morekeywords={foldr}]
x$_1$ ==> ([x$_1$, x$_2$, x$_3$], [x$_2$, x$_3$])
\end{lstlisting}
\end{frame}

\begin{frame}[fragile]{foldr is \emph{not} a silver bullet}
Let's get back to \texttt{dropWhile}:
\vspace{5mm}
\begin{lstlisting}[language=Haskell, morekeywords={foldr}]
dropWhile :: (a -> Bool) -> [a] -> [a]
dropWhile p = snd . foldr f ([], [])
  where
    f x (xs, rec) = (x : xs, if p x then rec else x : xs)
\end{lstlisting}
\pause
Compare to:
\begin{lstlisting}[language=Haskell, morekeywords={foldr}]
dropWhile :: (a -> Bool) -> [a] -> [a]
dropWhile p [] = []
dropWhile p (x:xs) = if p x then dropWhile p xs else x : xs
\end{lstlisting}
\vspace{5mm}
Programs written using \texttt{fold} can be less readable than programs written using explicit recursion.
\end{frame}

%%%%%%%%%%%%%%%%%%%%%%%%%%%%%%%%%%%%%%%%%%%%%%%%
\section{Iterators and recursors}
%%%%%%%%%%%%%%%%%%%%%%%%%%%%%%%%%%%%%%%%%%%%%%%%

\begin{frame}[fragile]{Iterators vs recursors}
\begin{lstlisting}[language=Haskell, morekeywords={foldr}]
foldr :: (a -> b -> b) -> b -> [a] -> b
foldr f v [] = v
foldr f v (x:xs) = f x (foldr f v xs)
\end{lstlisting}
\vspace{10mm}
\pause
\begin{lstlisting}[language=Haskell, morekeywords={foldr,foldrRec}]
foldrRec :: (a -> [a] -> b -> b) -> b -> [a] -> b
foldrRec f v [] = v
foldrRec f v (x:xs) = f x xs (foldrRec f v xs)
\end{lstlisting}
\vspace{5mm}
\texttt{foldr} is called an \emph{iterator} a.k.a \emph{catamorphism}\\
\texttt{foldrRec} is called a \emph{recursor} a.k.a \emph{paramorphism}\\
\texttt{foldrRec} implements a pattern called \emph{primitive recursion}
\end{frame}

\begin{frame}[fragile]{dropWhile again}
\begin{lstlisting}[language=Haskell, morekeywords={foldr,foldrRec}]
foldrRec :: (a -> [a] -> b -> b) -> b -> [a] -> b
foldrRec f v [] = v
foldrRec f v (x:xs) = f x xs (foldrRec f v xs)
\end{lstlisting}
\vspace{15mm}
\begin{lstlisting}[language=Haskell, morekeywords={foldr}]
dropWhile :: (a -> Bool) -> [a] -> [a]
dropWhile p = foldrRec
                (\x xs rec -> if p x then rec else x : xs)
                []
\end{lstlisting}
\end{frame}

%%%%%%%%%%%%%%%%%%%%%%%%%%%%%%%%%%%%%%%%%%%%%%%%
\section{Expressivity of fold}
%%%%%%%%%%%%%%%%%%%%%%%%%%%%%%%%%%%%%%%%%%%%%%%%

\begin{frame}[fragile]{foldrRec via foldr}
\begin{lstlisting}[language=Haskell, morekeywords={foldr,foldrRec}]
foldr :: (a -> b -> b) -> b -> [a] -> b
foldr f = foldrRec (const . f)
\end{lstlisting}
\vspace{15mm}
\pause
\begin{lstlisting}[language=Haskell, morekeywords={foldr,foldrRec}]
foldrRec :: (a -> [a] -> b -> b) -> b -> [a] -> b
foldrRec f v = snd . foldr g ([], v)
  where
    g x (xs, rec) = (x : xs, f x xs rec)
\end{lstlisting}
\end{frame}

\begin{frame}[fragile]{Definitions of foldl}
\begin{lstlisting}[language=Haskell, morekeywords={foldr,foldl}]
foldl :: (b -> a -> b) -> b -> [a] -> b
foldl f v [] = v
foldl f v (x:xs) = foldl f (f v x) xs
\end{lstlisting}
\pause
% The parentheses associate to the left
\begin{lstlisting}[language=Haskell, morekeywords={foldr}]
foldl ($\obackslash$) v [x$_1$, x$_2$, x$_3$, ..., x$_{n-1}$ x$_n$]
  =
(...(((v $\obackslash$ x$_1$) $\obackslash$ x$_2$) $\obackslash$ x$_3$) ... $\obackslash$ x$_{n-1}$) $\obackslash$ x$_n$
\end{lstlisting}
\pause
\hspace{4mm}$\mathrlap{\lefttorightarrow}\mid$
\begin{lstlisting}[language=Haskell, morekeywords={foldr}]
x$_n$ $\oslash$ (x$_{n-1}$ $\oslash$ ... (x$_3$ $\oslash$ (x$_2$ $\oslash$ (x$_1$ $\oslash$ v))) ... )
\end{lstlisting}
\pause
\begin{lstlisting}[language=Haskell, morekeywords={foldr}]
  =
(foldr (flip ($\obackslash$)) v) . reverse
\end{lstlisting}
\pause
\begin{lstlisting}[language=Haskell, morekeywords={foldr,foldl}]
foldl :: (b -> a -> b) -> b -> [a] -> b
foldl f v = (foldr (flip f) v) . reverse
\end{lstlisting}
%% reverse is expressible via fold, so right now we have expressed
%% foldl as a composition of two folds
\end{frame}

% now we can kind of merge reverse and list traversal, using deferred computation
% we are going to be simulating a stack
%% foldl :: (b -> a -> b) -> b -> [a] -> b
%% foldl f v xs = foldr (\x rec -> (\w -> rec (f w x))) id xs v
%% or equivalently
%% foldl f v xs = foldr (\x rec -> rec . flip f x) id xs v

\begin{frame}[fragile]{Definitions of foldl}
  %% (foldr (flip f) v) . (foldr snoc []) =
\begin{lstlisting}[language=Haskell, morekeywords={foldr,foldl}]
foldl ($\obackslash$) v xs
  =
(...(((v $\obackslash$ x$_1$) $\obackslash$ x$_2$) $\obackslash$ x$_3$) ... $\obackslash$ x$_{n-1}$) $\obackslash$ x$_n$
\end{lstlisting}
\pause
% employing our intuition we got from the reverse implementation
\begin{lstlisting}[language=Haskell, morekeywords={foldr,foldl}]
  =
($\obackslash$ x$_n$) . ($\obackslash$ x$_{n-1}$) ... . ($\obackslash$ x$_3$) . ($\obackslash$ x$_2$) . ($\obackslash$ x$_1$) $\textdollar$ v
\end{lstlisting}
\pause
% we lack the rightmost function which corresponds to the case of empty list
\begin{lstlisting}[language=Haskell, morekeywords={foldr,foldl}]
  =
id . ($\obackslash$ x$_n$) . ($\obackslash$ x$_{n-1}$) ... . ($\obackslash$ x$_3$) . ($\obackslash$ x$_2$) . ($\obackslash$ x$_1$) $\textdollar$ v
\end{lstlisting}
\pause
% composition is associative
\begin{lstlisting}[language=Haskell, morekeywords={foldr,foldl}]
  =
((...((id . ($\obackslash$ x$_n$)) . ($\obackslash$ x$_{n-1}$)) ... .
    ($\obackslash$ x$_3$)) . ($\obackslash$ x$_2$)) . ($\obackslash$ x$_1$) $\textdollar$ v
\end{lstlisting}
\pause
%% we got foldr written backwards, because now the last elements of the list are the innermost
\begin{lstlisting}[language=Haskell, morekeywords={foldr,foldl}]
  =
foldr (\x rec -> rec . ($\obackslash$ x)) id xs v
\end{lstlisting}
\end{frame}
%% Note that foldl is not of the form "foldr f v"

%% Further notes:
%% There are some useful ‘duality theorems’ concerning [fold] and [foldl],
%% and also guidelines for deciding which operator is best suited to particular applications in the "Introduction to Functional Programming using Haskell" book by Ruchard Bird
%% The Haskell community has produced a number of best practices, especially
%% with regard to laziness/strictness.

%% So far we have seen that foldr + tuples give greater expressivity
%% But in this talk we probably won't touch upon the fact that
%% foldr + higher-order functions gives us more that just primitive recursion --
%% we can express Ackermann's function in the higher-order setting

%%%%%%%%%%%%%%%%%%%%%%%%%%%%%%%%%%%%%%%%%%%%%%%%
\section{Universal Property and Category Theory}
%%%%%%%%%%%%%%%%%%%%%%%%%%%%%%%%%%%%%%%%%%%%%%%%

\begin{frame}[fragile]{Universal Property of fold}

\begin{theorem}
Given $f: a \rightarrow b \rightarrow b$, $v: b$, and $g: [a] \rightarrow b$, we have,
\begin{align*}
\begin{split}
g\;[\:] & = v \\
g\;(x:xs) & = f\;x\;(g\;xs)
\end{split}\quad
\Longleftrightarrow\quad
g = fold\;f\;v
\end{align*}
\end{theorem}

\end{frame}

\begin{frame}[fragile]{Universal Property of fold}

How to prove this?
\[
(+1)\;.\;sum = fold\;(+)\;1
\]
\pause
By universal property:
\begin{columns}
\begin{column}{0.5\textwidth}
\begin{align*}
((+1)\;.\;sum)\;[\;] & = 1 \\
((+1)\;.\;sum)\;(x : xs) & = (+)\;x\;(((+1)\;.\;sum)\;xs)
\end{align*}
\begin{align*}
sum\;[\;] + 1 & = 1 \\
sum\;(x : xs) + 1 & = x + (sum\;xs + 1)
\end{align*}
\end{column}
\begin{column}{0.5\textwidth}
\begin{align*}
g\;[\:] & = v \\
g\;(x:xs) & = f\;x\;(g\;xs)
\end{align*}
\end{column}
\end{columns}

\end{frame}

\begin{frame}[fragile]{Universal Property of fold: List objects}
%Let $C$ be a category with a terminal object $1$ and binary products.

%A \alert{list object} $L(A)$ over some object $A$, consists of two arrows,
%\[
%nil: 1 \rightarrow L(A)
%cons: A \times L(A) \rightarrow L(A)
%\]
%such that for any other object $B$ and two arrows,
%\[
%v: 1 \rightarrow B
%f: A \times B \rightarrow B
%\]
%there is a unique arrow $fold(f,v): L(A) \rightarrow B$ making the following diagrams commute:
\centering
\begin{tikzcd}
1 \arrow[r, "nil" above] & L(A) & A \times L(A) \arrow[r, "cons" above] & L(A)
\end{tikzcd}

\begin{tikzcd}
1 \arrow[r, "v" above] & B & A \times B \arrow[r, "f" above] & B
\end{tikzcd}

\begin{tikzcd}[column sep=huge]
L(A) \arrow[r, "\exists !\;fold\;f\;v" above, dashed] & B
\end{tikzcd}

\begin{columns}
\begin{column}{0.5\textwidth}
\centering
\begin{tikzcd}
1 \arrow[rd, "v\quad" left, near end] \arrow[r, "nil" above] & L(A) \arrow[d, "fold\;f\;v" right] \\
& B
\end{tikzcd}

\[
(fold\;f\;v) \circ nil = v
\]
\end{column}
\begin{column}{0.5\textwidth}
\begin{tikzcd}
A \times L(A) \arrow[d,"id_A \times (fold\;f\;v)" left] \arrow[r, "cons" above] & L(A) \arrow[d, "fold\;f\;v" right] \\
A \times B \arrow[r, "f" below] & B
\end{tikzcd}

\begin{align*}
(fold\;f\;v) \circ cons = f \circ (id_A \times fold\;f\;v) \\
(fold\;f\;v)\;(cons\;(x, l)) = \\ 
f\;((id_A \times fold\;f\;v) (x, l)) = \\
f\;(x, fold\;f\;v\;l)
\end{align*}

\end{column}
\end{columns}
\end{frame}

\begin{frame}[fragile]{Universal Property of fold: Initial Algebras}

\begin{columns}
\begin{column}{0.5\textwidth}
\begin{align*}
F_A\;X & :\equiv 1 + A \times X \\
map\;F_A\;f & :\equiv id_1 + id_A \times f
\end{align*}
\end{column}
\begin{column}{0.5\textwidth}
\begin{tikzcd}
X \arrow[dd, "f" left] & 1 + A \times X \arrow[dd, "id_1 + id_A \times f" right]\\
\  \arrow[r, "map\;F_A\;f", mapsto] & \  \\
Y & 1 + A \times Y
\end{tikzcd}
\end{column}
\end{columns}

\centering
\begin{tikzcd}
F_A\;L(A) \arrow[r, "In" above] & L(A)
\end{tikzcd}

\begin{tikzcd}
F_A\;B \arrow[r, "f" above] & B
\end{tikzcd}

\begin{tikzcd}[column sep=huge]
L(A) \arrow[r, "\exists !\;cata\;f" above, dashed] & B
\end{tikzcd}

\begin{columns}
\begin{column}{0.5\textwidth}
\begin{tikzcd}
F_A\;L(A) \arrow[d,"map\;F_A\;(cata\;f)" left] \arrow[r, "In" above] & L(A) \arrow[d, "cata\;f" right] \\
F_A\;B \arrow[r, "f" below] & B
\end{tikzcd}
\end{column}
\begin{column}{0.5\textwidth}
\begin{align*}
(cata\;f) \circ In & = f \circ (map\;F_A\;(cata\;f)) \\
                   & = f \circ (id_1 + id_A \times (cata\;f))
\end{align*}
\end{column}
\end{columns}

% You can obtain the equations for the fold of your datatype by changing the functor
% and extracting the equations from the general commutative diagram for the initial algebra
\end{frame}


%%%%%%%%%%%%%%%%%%%%%%%%%%%%%%%%%%%%%%%%%%%%%%%%
%% \section{Generalization to other datatypes}
%%%%%%%%%%%%%%%%%%%%%%%%%%%%%%%%%%%%%%%%%%%%%%%%

\begin{frame}[fragile]{Fold and other datatypes}
% But there is an easier way to obtain how the fold (or catamorphism) for your datatype should look like.
% From the universal property for fold, we see that if we apply fold with the
% constructors for the list, the universal property says that the fold is equal
% to the identity function.
% So, a general rule of what a catamorphism should do is that it must behave like the identity
% function when applied on the constructors.

\begin{align*}
\begin{split}
id\;[\:] & = [\:] \\
id\;(x:xs) & = (:)\;x\;(id\;xs)
\end{split}\quad
\Longleftrightarrow\quad
id = fold\;(:)\;[\:]
\end{align*}
\pause
% 1) A fold or catamorphism should pattern match on the constructors of the datatype and resolve base cases by providing constants as result.
% 2) For non-base cases, it must recurse on the subcomponents.
% 3) At the end, it must apply a function to aggregate all the results.

\begin{lstlisting}[language=Haskell, morekeywords={foldr}]
foldr :: (a -> b -> b) -> b -> [a] -> b
foldr f v [] = v
foldr f v (x:xs) = f x (foldr f v xs)
\end{lstlisting}

\end{frame}

\begin{frame}[fragile]{Fold (catamorphism) and trees}
\begin{lstlisting}[language=Haskell]
data LTree a = 
   Leaf a | Split (LTree a) (LTree a)
data BTree a = 
   Empty | Node a (BTree a) (BTree a)
\end{lstlisting}
\pause

%1) The catamorphism consumes something of the datatype into an arbitrary type
\begin{lstlisting}[language=Haskell, morekeywords={cata_LTree, cata_BTree}]
cata_LTree :: ....... -> LTree a -> b
cata_BTree :: ....... -> BTree a -> b
\end{lstlisting}
\pause

%2) Each parameter corresponds to a constructor, where each occurrence of the datatype in the constructor is replaced by the target type b
\begin{lstlisting}[language=Haskell, morekeywords={cata_LTree, cata_BTree}]
cata_LTree :: (b -> b -> b) -> (a -> b) -> LTree a -> b
cata_BTree :: (a -> b -> b -> b) -> b -> BTree a -> b
\end{lstlisting}

Since:

\begin{lstlisting}[language=Haskell]
Split :: LTree a -> LTree a -> LTree a
Leaf :: a -> LTree a
Node :: a -> BTree a -> BTree a -> BTree a
Empty :: BTree a
\end{lstlisting}
\end{frame}

\begin{frame}[fragile]{Fold (catamorphism) and trees}
\begin{lstlisting}[language=Haskell, morekeywords={cata_LTree}]
cata_LTree :: (b -> b -> b) -> (a -> b) -> LTree a -> b
cata_LTree fs fl (Leaf x) = fl x
cata_LTree fs fl (Split l r) = 
  fs (cata_LTree fs fl l) (cata_LTree fs fl r)
\end{lstlisting}

\begin{lstlisting}[language=Haskell, morekeywords={cata_BTree}]
cata_BTree :: (a -> b -> b -> b) -> b -> BTree a -> b
cata_BTree fn v Empty = v
cata_BTree fn v (Node x l r) = 
  fn x (cata_BTree fn v l) (cata_BTree fn v r)
\end{lstlisting}

\end{frame}

%%%%%%%%%%%%%%%%%%%%%%%%%%%%%%%%%%%%%%%%%%%%%%%%
%\section{Recursion Schemes}
%%%%%%%%%%%%%%%%%%%%%%%%%%%%%%%%%%%%%%%%%%%%%%%%

%% "Sorting morphisms" by L. Augusteijn(1999)
%% - insertion sort -- catamorphism over list
%% - merge sort can be expressed as a morphism over the leaf tree data type
%% - quick sort and heap sort can be expressed as morphisms over the binary tree data type

% This is what a catamorphism does in general: replacing the constructors of a data type by other functions.

%\begin{frame}[fragile]{Catamorphisms}
%We already know that \texttt{fold} is a list catamorphism.
% just a reminder and also the following style will feel more familiar
%\begin{lstlisting}[language=Haskell]
%data List a =
%  Nil |
%  Cons a (List a)
%\end{lstlisting}
%\pause
%\begin{lstlisting}[language=Haskell]
%list_cata :: u -> (a -> u -> u) -> List a -> u
%list_cata v fc = cata where
%  cata Nil = v
%  cata (Cons x xs) = fc x (cata xs)
%\end{lstlisting}
%\end{frame}

%\begin{frame}[fragile]{Catamorphisms}
%\begin{lstlisting}[language=Haskell]
%data LfTree x =
%   Leaf x
% | Split (LeafTree x) (LfTree x)
%\end{lstlisting}
%\pause
%\begin{lstlisting}[language=Haskell]
%leaftree_cata :: (x -> u) -> (u -> u -> u) -> LfTree x -> u
%leaftree_cata fl fs = cata where
%  cata (Leaf x) = fl x
%  cata (Split l r) = fs (cata l) (cata r)
%\end{lstlisting}
%\end{frame}

%\begin{frame}[fragile]{Catamorphisms}
%\begin{lstlisting}[language=Haskell]
%data BinTree x =
%  Tip |
%  Branch x (BinTree x) (BinTree x)
%\end{lstlisting}
%\pause
%\begin{lstlisting}[language=Haskell]
%bintree_cata :: u -> (x -> u -> u -> u) -> BinTree x -> u
%bintree_cata v f = cata where
%  cata Tip = v
%  cata (Branch x l r) = f x (cata l) (cata r)
%\end{lstlisting}
%\end{frame}

%% all the morphisms should be generated by the compiler


%% \item catamorphism (Greek: κατά "downwards" and μορφή "form, shape") \\
%% A \emph{catamorphism} on a type T is a function of type T → U that destruct and object of type T according to the structure of T,
%% calls itself recursively of any components of T that are also of type T and
%% combines this recursive result with the remaining components of T to a U.

% ?? insert the picture from Sorting morphisms (page 3)
% It can be observed that this catamorphism replaces the empty list by a and the non-empty list constructor (:) by f.



%%% Probably we should move these into their respective slides
%% \begin{frame}[fragile]{Terminology}
%% \Large
%% \begin{itemize}
  %% \item TODO
  %% \item catamorphism (Greek: κατά "downwards" and μορφή "form, shape") \\
  %% \item anamorphism (ἀνά = "upwards") \\
  %% \item hylomorphism (ὕλη = "wood, matter") \\
  %% \item paramorphism (παρά = "close together")
%% \end{itemize}
%% \end{frame}

%% \begin{frame}[fragile]{Sorting algorithms}
%% Fixing a recursion scheme underlying an implementation can give us a particular sorting algorithm
%% See Sorting morphims in the links file
%% \Large
%% \begin{itemize}
  %% \item The value of this approach is not so much in obtaining a nice implementation of some algorithm, but in unraveling its structure.
%% \end{itemize}
%% \end{frame}

\begin{frame}[fragile]{Compiler optimizations: rewrite rules}
\begin{lstlisting}[language=Haskell, morekeywords={foldr}]
{-# RULES
  "map" [~1] forall f xs.
             map f xs =
             build (\c n -> foldr (mapFB c f) n xs)
-}
\end{lstlisting}

where
\begin{lstlisting}[language=Haskell, morekeywords={foldr}]
build :: (forall b. (a -> b -> b) -> b -> b) -> [a]
build g = g (:) []

mapFB :: (e -> lst -> lst) -> (a -> e) -> a -> lst -> lst
mapFB c f = \x ys -> c (f x) ys
\end{lstlisting}
%% the above code is taken from
%% http://git.haskell.org/ghc.git (libraries/base/GHC/Base.hs)

The equations guiding the compiler must be correct!
\end{frame}

\begin{frame}[fragile]{Advantage: Reasoning made easier!}
Interactive session in Coq: the fusion property of \texttt{fold}.
\end{frame}

\begin{frame}[fragile]{Outro}
\huge
Thank you!

\vspace{10mm}

Get in touch with us after the talk!

\vspace{10mm}
\Large
\href{https://github.com/anton-trunov/fold-tutorial-talk}{github.com/anton-trunov/fold-tutorial-talk}

\end{frame}

\begin{frame}[fragile]{Outro}
\Huge
Questions?
\end{frame}
\end{document}
